

%\begin{document}

Neste trabalho apresentamos a descrição da tecnologia RFID, explicando como funcionam os seus diferentes componentes: tags, leitores e \textit{Middleware}. Do ponto de vista de integração de dados, o \textit{Middleware} é o elemento mais importante, uma vez que ele é responsável pela interface entre os sistemas físicos de acquisição de dados RFID e os outros sistemas empresariais. O RFID é usado principalmente para a localização e monitoramento de ativos, onde precisamos saber por onde e quando passou cada elemento da cadeia produtiva. Por causa disso, esta é uma tecnlogia que está em constante aprimoramento e crescimento no número de empresas que a adotam.

Através dos estudos de casos, foi possível ver na prática como os conceitos explicados anteriormente são aplicados, mostrando os desafios e dificuldades encontradas na implementação deste tipo de sistema. Os mesmos problemas decorrentes da heterogeneidade dos elementos da arquitetura do sistema puderam ser vistos nestas aplicações práticas. Para tentar contorná-los, forma criados diversas normas e entre elas podemos citar a EPCGlobal. Ela define tanto os aspectos de hardware quanto os da arquitetura, de forma melhorar visivelmente a eficiência de todo do sistema. Isso foi comprovado pelo estudo de caso da norma apresentado.

Logo, podemos concluir que o trabalho realizado cumpriu os objetivos propostos, apresentando uma boa descrição da tecnologia RFID, focada nos aspectos de integração de dados.  

 
%\end{document}